%%
% Copyright (c) 2017 - 2019, Pascal Wagler;
% Copyright (c) 2014 - 2019, John MacFarlane
%
% All rights reserved.
%
% Redistribution and use in source and binary forms, with or without
% modification, are permitted provided that the following conditions
% are met:
%
% - Redistributions of source code must retain the above copyright
% notice, this list of conditions and the following disclaimer.
%
% - Redistributions in binary form must reproduce the above copyright
% notice, this list of conditions and the following disclaimer in the
% documentation and/or other materials provided with the distribution.
%
% - Neither the name of John MacFarlane nor the names of other
% contributors may be used to endorse or promote products derived
% from this software without specific prior written permission.
%
% THIS SOFTWARE IS PROVIDED BY THE COPYRIGHT HOLDERS AND CONTRIBUTORS
% "AS IS" AND ANY EXPRESS OR IMPLIED WARRANTIES, INCLUDING, BUT NOT
% LIMITED TO, THE IMPLIED WARRANTIES OF MERCHANTABILITY AND FITNESS
% FOR A PARTICULAR PURPOSE ARE DISCLAIMED. IN NO EVENT SHALL THE
% COPYRIGHT OWNER OR CONTRIBUTORS BE LIABLE FOR ANY DIRECT, INDIRECT,
% INCIDENTAL, SPECIAL, EXEMPLARY, OR CONSEQUENTIAL DAMAGES (INCLUDING,
% BUT NOT LIMITED TO, PROCUREMENT OF SUBSTITUTE GOODS OR SERVICES;
% LOSS OF USE, DATA, OR PROFITS; OR BUSINESS INTERRUPTION) HOWEVER
% CAUSED AND ON ANY THEORY OF LIABILITY, WHETHER IN CONTRACT, STRICT
% LIABILITY, OR TORT (INCLUDING NEGLIGENCE OR OTHERWISE) ARISING IN
% ANY WAY OUT OF THE USE OF THIS SOFTWARE, EVEN IF ADVISED OF THE
% POSSIBILITY OF SUCH DAMAGE.
%%

%%
% This is the Eisvogel pandoc LaTeX template.
%
% For usage information and examples visit the official GitHub page:
% https://github.com/Wandmalfarbe/pandoc-latex-template
%%

% Options for packages loaded elsewhere
\PassOptionsToPackage{unicode}{hyperref}
\PassOptionsToPackage{hyphens}{url}
\PassOptionsToPackage{dvipsnames,svgnames*,x11names*,table}{xcolor}
%
\documentclass[
  a4paper,
,tablecaptionabove
]{scrartcl}
\usepackage{lmodern}
\usepackage{setspace}
\setstretch{1.2}
\usepackage{amssymb,amsmath}
\usepackage{ifxetex,ifluatex}
\ifnum 0\ifxetex 1\fi\ifluatex 1\fi=0 % if pdftex
  \usepackage[T1]{fontenc}
  \usepackage[utf8]{inputenc}
  \usepackage{textcomp} % provide euro and other symbols
\else % if luatex or xetex
  \usepackage{unicode-math}
  \defaultfontfeatures{Scale=MatchLowercase}
  \defaultfontfeatures[\rmfamily]{Ligatures=TeX,Scale=1}
\fi
% Use upquote if available, for straight quotes in verbatim environments
\IfFileExists{upquote.sty}{\usepackage{upquote}}{}
\IfFileExists{microtype.sty}{% use microtype if available
  \usepackage[]{microtype}
  \UseMicrotypeSet[protrusion]{basicmath} % disable protrusion for tt fonts
}{}
\makeatletter
\renewcommand{\verbatim@font}{\ttfamily\scriptsize}
\makeatother
\makeatletter
\@ifundefined{KOMAClassName}{% if non-KOMA class
  \IfFileExists{parskip.sty}{%
    \usepackage{parskip}
  }{% else
    \setlength{\parindent}{0pt}
    \setlength{\parskip}{6pt plus 2pt minus 1pt}}
}{% if KOMA class
  \KOMAoptions{parskip=half}}
\makeatother
\usepackage{xcolor}
\definecolor{default-linkcolor}{HTML}{A50000}
\definecolor{default-filecolor}{HTML}{A50000}
\definecolor{default-citecolor}{HTML}{4077C0}
\definecolor{default-urlcolor}{HTML}{4077C0}
\IfFileExists{xurl.sty}{\usepackage{xurl}}{} % add URL line breaks if available
\IfFileExists{bookmark.sty}{\usepackage{bookmark}}{\usepackage{hyperref}}
\hypersetup{
  pdftitle={Técnicas de soft computing para Aprendizaje y Optimización},
  pdfauthor={Francisco Luque Sánchez},
  colorlinks=true,
  linkcolor=default-linkcolor,
  filecolor=default-filecolor,
  citecolor=default-citecolor,
  urlcolor=blue,
  breaklinks=true,
  pdfcreator={LaTeX via pandoc with the Eisvogel template}}
\urlstyle{same} % disable monospaced font for URLs
\usepackage[margin=2.5cm,includehead=true,includefoot=true,centering,]{geometry}
\usepackage{longtable,booktabs}
% Correct order of tables after \paragraph or \subparagraph
\usepackage{etoolbox}
\makeatletter
\patchcmd\longtable{\par}{\if@noskipsec\mbox{}\fi\par}{}{}
\makeatother
% Allow footnotes in longtable head/foot
\IfFileExists{footnotehyper.sty}{\usepackage{footnotehyper}}{\usepackage{footnote}}
\makesavenoteenv{longtable}
% add backlinks to footnote references, cf. https://tex.stackexchange.com/questions/302266/make-footnote-clickable-both-ways
\usepackage{footnotebackref}
\usepackage{graphicx,grffile}
\makeatletter
\def\maxwidth{\ifdim\Gin@nat@width>\linewidth\linewidth\else\Gin@nat@width\fi}
\def\maxheight{\ifdim\Gin@nat@height>\textheight\textheight\else\Gin@nat@height\fi}
\makeatother
% Scale images if necessary, so that they will not overflow the page
% margins by default, and it is still possible to overwrite the defaults
% using explicit options in \includegraphics[width, height, ...]{}
\setkeys{Gin}{width=\maxwidth,height=\maxheight,keepaspectratio}
\setlength{\emergencystretch}{3em}  % prevent overfull lines
\providecommand{\tightlist}{%
  \setlength{\itemsep}{0pt}\setlength{\parskip}{0pt}}
\setcounter{secnumdepth}{3}

% Make use of float-package and set default placement for figures to H.
% The option H means 'PUT IT HERE' (as  opposed to the standard h option which means 'You may put it here if you like').
\usepackage{float}
\floatplacement{figure}{H}


\title{Técnicas de soft computing para Aprendizaje y Optimización}
\usepackage{etoolbox}
\makeatletter
\providecommand{\subtitle}[1]{% add subtitle to \maketitle
  \apptocmd{\@title}{\par {\large #1 \par}}{}{}
}
\makeatother
\subtitle{Algoritmos bionispirados}
\author{Francisco Luque Sánchez}
\date{02/04/2020}



%%
%% added
%%

%
% language specification
%
% If no language is specified, use English as the default main document language.
%

\ifnum 0\ifxetex 1\fi\ifluatex 1\fi=0 % if pdftex
  \usepackage[shorthands=off,main=english]{babel}
\else
    % Workaround for bug in Polyglossia that breaks `\familydefault` when `\setmainlanguage` is used.
  % See https://github.com/Wandmalfarbe/pandoc-latex-template/issues/8
  % See https://github.com/reutenauer/polyglossia/issues/186
  % See https://github.com/reutenauer/polyglossia/issues/127
  \renewcommand*\familydefault{\sfdefault}
    % load polyglossia as late as possible as it *could* call bidi if RTL lang (e.g. Hebrew or Arabic)
  \usepackage{polyglossia}
  \setmainlanguage[]{english}
\fi



%
% for the background color of the title page
%
\usepackage{pagecolor}
\usepackage{afterpage}
\usepackage{tikz}
\usepackage[margin=2.5cm,includehead=true,includefoot=true,centering]{geometry}

%
% break urls
%
\PassOptionsToPackage{hyphens}{url}

%
% When using babel or polyglossia with biblatex, loading csquotes is recommended
% to ensure that quoted texts are typeset according to the rules of your main language.
%
\usepackage{csquotes}

%
% captions
%
\definecolor{caption-color}{HTML}{777777}
\usepackage[font={stretch=1.2}, textfont={color=caption-color}, position=top, skip=4mm, labelfont=bf, singlelinecheck=false, justification=raggedright]{caption}
\setcapindent{0em}

%
% blockquote
%
\definecolor{blockquote-border}{RGB}{221,221,221}
\definecolor{blockquote-text}{RGB}{119,119,119}
\usepackage{mdframed}
\newmdenv[rightline=false,bottomline=false,topline=false,linewidth=3pt,linecolor=blockquote-border,skipabove=\parskip]{customblockquote}
\renewenvironment{quote}{\begin{customblockquote}\list{}{\rightmargin=0em\leftmargin=0em}%
\item\relax\color{blockquote-text}\ignorespaces}{\unskip\unskip\endlist\end{customblockquote}}

%
% Source Sans Pro as the de­fault font fam­ily
% Source Code Pro for monospace text
%
% 'default' option sets the default
% font family to Source Sans Pro, not \sfdefault.
%
\ifnum 0\ifxetex 1\fi\ifluatex 1\fi=0 % if pdftex
    \usepackage[default]{sourcesanspro}
  \usepackage{sourcecodepro}
  \else % if not pdftex
    \usepackage[default]{sourcesanspro}
  \usepackage{sourcecodepro}

  % XeLaTeX specific adjustments for straight quotes: https://tex.stackexchange.com/a/354887
  % This issue is already fixed (see https://github.com/silkeh/latex-sourcecodepro/pull/5) but the
  % fix is still unreleased.
  % TODO: Remove this workaround when the new version of sourcecodepro is released on CTAN.
  \ifxetex
    \makeatletter
    \defaultfontfeatures[\ttfamily]
      { Numbers   = \sourcecodepro@figurestyle,
        Scale     = \SourceCodePro@scale,
        Extension = .otf }
    \setmonofont
      [ UprightFont    = *-\sourcecodepro@regstyle,
        ItalicFont     = *-\sourcecodepro@regstyle It,
        BoldFont       = *-\sourcecodepro@boldstyle,
        BoldItalicFont = *-\sourcecodepro@boldstyle It ]
      {SourceCodePro}
    \makeatother
  \fi
  \fi

%
% heading color
%
\definecolor{heading-color}{RGB}{40,40,40}
\addtokomafont{section}{\color{heading-color}}
% When using the classes report, scrreprt, book,
% scrbook or memoir, uncomment the following line.
%\addtokomafont{chapter}{\color{heading-color}}

%
% variables for title and author
%
\usepackage{titling}
\title{Técnicas de soft computing para Aprendizaje y Optimización}
\author{Francisco Luque Sánchez}

%
% tables
%

\definecolor{table-row-color}{HTML}{F5F5F5}
\definecolor{table-rule-color}{HTML}{999999}

%\arrayrulecolor{black!40}
\arrayrulecolor{table-rule-color}     % color of \toprule, \midrule, \bottomrule
\setlength\heavyrulewidth{0.3ex}      % thickness of \toprule, \bottomrule
\renewcommand{\arraystretch}{1.3}     % spacing (padding)


%
% remove paragraph indention
%
\setlength{\parindent}{0pt}
\setlength{\parskip}{6pt plus 2pt minus 1pt}
\setlength{\emergencystretch}{3em}  % prevent overfull lines

%
%
% Listings
%
%


%
% header and footer
%
\usepackage{fancyhdr}

\fancypagestyle{eisvogel-header-footer}{
  \fancyhead{}
  \fancyfoot{}
  \lhead[02/04/2020]{Técnicas de soft computing para Aprendizaje y Optimización}
  \chead[]{}
  \rhead[Técnicas de soft computing para Aprendizaje y Optimización]{02/04/2020}
  \lfoot[\thepage]{Francisco Luque Sánchez}
  \cfoot[]{}
  \rfoot[Francisco Luque Sánchez]{\thepage}
  \renewcommand{\headrulewidth}{0.4pt}
  \renewcommand{\footrulewidth}{0.4pt}
}
\pagestyle{eisvogel-header-footer}

%%
%% end added
%%

\begin{document}

%%
%% begin titlepage
%%
\begin{titlepage}
\newgeometry{top=2cm, right=4cm, bottom=3cm, left=4cm}
\tikz[remember picture,overlay] \node[inner sep=0pt] at (current page.center){\includegraphics[width=\paperwidth,height=\paperheight]{background.pdf}};
\newcommand{\colorRule}[3][black]{\textcolor[HTML]{#1}{\rule{#2}{#3}}}
\begin{flushleft}
\noindent
\\[-1em]
\color[HTML]{5F5F5F}
\makebox[0pt][l]{\colorRule[435488]{1.3\textwidth}{4pt}}
\par
\noindent

% The titlepage with a background image has other text spacing and text size
{
  \setstretch{2}
  \vfill
  \vskip -8em
  \noindent {\huge \textbf{\textsf{Técnicas de soft computing para Aprendizaje y Optimización}}}
    \vskip 1em
  {\Large \textsf{Algoritmos bionispirados}}
    \vskip 2em
  \noindent {\Large \textsf{Francisco Luque Sánchez} \vskip 0.6em \textsf{02/04/2020}}
  \vfill
}


\end{flushleft}
\end{titlepage}
\restoregeometry

%%
%% end titlepage
%%



\hypertarget{introducciuxf3n}{%
\section{Introducción}\label{introducciuxf3n}}

En este trabajo se va a realizar un pequeño estudio comparativo entre
tres modelos distintos de algoritmos bioinspirados. Los modelos que
estudiaremos son los algoritmos genéticos (GA), los algoritmos de
optimización basados en enjambres (PSO), y los algoritmos de
optimización por colonia de abejas (ABC). Estos tres modelos forman
parte, como hemos dicho anteriormente, de lo que se conocen como
algoritmos bioinspirados. Dichas metaheurísticas son aquellas que tratan
de encontrar soluciones a problemas, normalmente de optimización, en las
que la búsqueda de soluciones óptimas de forma exacta no es viable en
términos de recursos, y por tanto, recurren a inspiraciones basadas en
sistemas naturales para guiar el proceso de búsqueda de soluciones,
sacrificando la optimalidad de la solución a cambio de cierta
eficiencia. De esta forma, se consiguen soluciones razonablemente buenas
al problema en cuestión en un tiempo asumible.

La principal diferencia entre estos tres paradigmas radica en el sistema
biológico en el que toman su inspiración para la búsqueda de soluciones.
En todos los casos, se estudia el comportamiento de un sistema biológico
complejo y se modela matemáticamente su comportamiento en términos de la
evolución del mismo. En este caso:

\begin{itemize}
\tightlist
\item
  GA: La inspiración de estos algoritmos viene de la teoría de la
  evolución Darwiniana.
\item
  PSO: Estos algoritmos toman como inspiración el comportamiento de de
  animales que forman colonias de individuos
\item
  ABC: La inspiración viene del comportamiento de los enjambres de
  abejas a la hora de recolectar comida
\end{itemize}

A continuación, daremos una descripción más precisa de cada uno de los
modelos.

\hypertarget{resumen-del-funcionamiento-de-los-modelos}{%
\section{Resumen del funcionamiento de los
modelos}\label{resumen-del-funcionamiento-de-los-modelos}}

En este apartado veremos cómo funcionan los modelos nombrados
anteriormente.

\hypertarget{algoritmos-genuxe9ticos}{%
\subsection{Algoritmos genéticos}\label{algoritmos-genuxe9ticos}}

Este tipo de modelos están inspirados en la teoría de la evolución
propuesta por Darwin. Dado un problema de optimización, el algoritmo
genético propone considerar una población de soluciones, de forma que
cada individuo de la población represente una solución al problema a
resolver. De esta forma, en sucesivas iteraciones, los individuos de la
población compiten entre sí para generar descendencia, de forma que
cuanto mejores sean las soluciones al problema, más probable será que se
reproduzcan o sobrevivan entre dos iteraciones del algoritmo. Además,
aleatoriamente pueden aparecer mutaciones en ciertos individuos de la
población.

Concretamente, independientemente del problema a resolver, el flujo de
un algoritmo genético para resolver un problema es más o menos estándar.
Dado un problema de optimización, en el que supondremos que podemos
evaluar la calidad de una solución a partir de una función \(f\) (que en
la mayoría de los casos será la función a optimizar directamente, o
acompañada de algún término de penalización):

\begin{itemize}
\tightlist
\item
  Se parte de una población de soluciones inicial (la cual puede ser
  generada aleatoriamente o mediante alguna política establecida de
  antemano)
\item
  Mientras no se haya cumplido la condición de parada (usualmente,
  número de etapas del algoritmo, o número de evaluaciones de la función
  objetivo \(f\)):

  \begin{itemize}
  \tightlist
  \item
    Se evalúan los individuos de la población
  \item
    Se establece un criterio de selección de padres para generar la
    población de hijos
  \item
    Se genera la población de hijos a partir de los padres utilizando
    una estrategia de cruce
  \item
    Se producen mutaciones en la población
  \item
    A través de una estrategia de supervivencia, se seleccionan los
    individuos que conformarán la siguiente generación de padres
    (supervivientes)
  \end{itemize}
\end{itemize}

Necesitamos, por tanto, definir los siguientes elementos para definir
por completo un algoritmo genético:

\begin{itemize}
\tightlist
\item
  La función de evaluación, la cual recibe como entrada una solución y
  devuelve su calidad. Suele coincidir con la función a optimizar.
\item
  Codificación de la solución. Hace referencia a cómo se representa y se
  interpreta una solución al problema. Cada uno de los individuos de
  nuestra población vendrá determinado por una de estas codificaciones
\item
  Criterio de selección. Determina cómo se seleccionan los padres a la
  hora de generar la población de hijos. Algunos ejemplos típicos son el
  torneo binario o la ruleta ponderada por la bondad de la solución
\item
  Operador de cruce. Indica cómo se deben mezclar dos soluciones padres
  para dar lugar a un hijo.
\item
  Operador de mutación. Indica cómo se deben alterar los elementos de la
  población individualmente. Asociada a esta función, también hay que
  determinar la tasa de mutación, es decir, cuán probable es que se
  produzcan mutaciones en la población
\item
  Estrategia de supervivencia. Especifica cómo se conforma la población
  de la siguiente época a partir de la población anterior y los hijos
  generados.
\end{itemize}

\hypertarget{optimizaciuxf3n-basada-en-enjambres}{%
\subsection{Optimización basada en
enjambres}\label{optimizaciuxf3n-basada-en-enjambres}}

Pasamos a comentar la optimización basada en enjambres (Particle Swarm
Optimization, o PSO en inglés). Este conjunto de algoritmos toma como
inspiración el comportamiento de animales que se organizan en enjambres
para tratar de optimizar una función. En particular, estos algoritmos
son ampliamente utilizados para la optimización de funciones de variable
real.

Este algoritmo ha sufrido distintas modificaciones a lo largo del
tiempo, que trataban de mejorar la propuesta original. El funcionamiento
básico del algoritmo es el siguiente:

\begin{itemize}
\tightlist
\item
  Se inicializa aleatoriamente la población de soluciones, las cuales
  suelen venir representadas por un punto del espacio \(\mathbb{R}^n\),
  y una velocidad.
\item
  Se evalúa la calidad de dichas soluciones a través de la función a
  optimizar.
\item
  Se calcula la mejor posición hasta el momento de cada partícula.
\item
  Se calcula la mejor solución del conjunto completo.
\item
  Se actualiza la posición y la velocidad de las partículas en función
  de su mejor posición, la mejor posición global, y las ecuaciones de
  actualización.
\end{itemize}

Las sucesivas modificaciones que ha ido sufriendo el algoritmo han sido
prácticamente en su totalidad en las ecuaciones que se utilizan para
actualizar la posición y la velocidad. En una primera instancia, las
ecuaciones de movimiento eran bastante simples. La velocidad se
actualizaba en cada iteración teniendo en cuenta exclusivamente la mejor
posición global, de forma que se apuntaba el vector director en dicha
dirección, y la nueva posición se calculaba sumando a la posición
anterior la velocidad actual. El problema que surgía en este caso era la
falta de capacidad de exploración del algoritmo, que convergía
prematuramente. En sucesivas actualizaciones del modelo, se añadieron
términos de inercia, los cuales favorecen que las partículas mantengan
su dirección de movimiento, así como términos que modificaban la
velocidad en función de la mejor posición de la propia partícula hasta
el momento, no sólo a partir de la mejor posición global. De esta forma,
existen tres fuentes de variabilidad para el cambio de velocidad de cada
partícula. La ecuación de actualización de la posición es esencialmente
la misma desde la propuesta original. Una vez tenemos la ecuación de
actualización de la velocidad, podemos añadir tres constantes
multiplicativas, que representan la importancia que tiene en nuestro
modelo la componente inercial, la componente individual (dirección hasta
nuestra mejor posición) y la componente social (dirección hacia la mejor
posición global). Además, se introducen dos valores aleatorios, en las
componentes social e individual, que se modifican en cada iteración y
que añaden variabilidad al sistema (también multiplicativas).
Finalmente, aparece una constante de escalado, la cual podemos asemejar
a la tasa de aprendizaje de los modelos basados en gradiente, la cual
hace de factor de escala para todo el vector velocidad, y que va
decreciendo en etapas tardías del algoritmo, para hacer que el
movimiento en etapas tardías sea más lento, con la finalidad de explotar
los vecindarios de las buenas soluciones al final del proceso de
búsqueda.

\hypertarget{optimizaciuxf3n-por-colonia-de-abejas}{%
\subsection{Optimización por colonia de
abejas}\label{optimizaciuxf3n-por-colonia-de-abejas}}

Este último grupo de algoritmos tiene su inspiración en un tipo
particular de colonia. La inspiración de el modelo de colonia de abejas
(Artificial Bee Colony, o ABC) aparece del estudio de la conducta de las
colmenas de abejas a la hora de recolectar comida. Podría decirse que
este algoritmo es una especialización del modelo anterior. En este caso,
tenemos un conjunto de partículas en el espacio de soluciones que
simulan el comportamiento de las abejas. Tendremos, por tanto, tres
tipos distintos de partículas en el espacio:

\begin{itemize}
\tightlist
\item
  Buscadoras: Estas partículas estarán dedicadas a localizar fuentes de
  comida en el espacio de búsqueda (puntos en los que la función toma
  valores bajos o altos, en función de si estamos resolviendo un
  problema de minimización o maximización)
\item
  Recolectoras: Dedicadas a explotar las fuentes de alimento localizadas
  por las buscadoras (se dedican a hacer búsquedas locales en el
  vecindario de las buenas soluciones obtenidas, tratando de encontrar
  soluciones mejores en el entorno)
\item
  Observadoras: Partículas sin actividad, las cuales se convertirán en
  recolectoras a partir de la observación de otras recolectoras y sus
  indicaciones para buscar las fuentes de alimento
\end{itemize}

Para tratar de asemejarse lo más posible al comportamiento de los
enjambres, existe una comunicación entre los distintos tipos de abejas.
El algoritmo transcurre en tres fases que se repiten hasta que se
produzca la condición de parada del algoritmo:

\begin{itemize}
\tightlist
\item
  Fase de recolectoras: Las abejas recolectoras buscan fuentes de néctar
  más abundantes que se encuentren cercanas a la fuente de néctar en la
  que se encuentran (búsqueda local)
\item
  Fase de observadoras: Las abejas observadoras eligen
  probabilísticamente dirigirse a recolectar (se convierten en
  recolectoras) a los distintos focos de alimento en función de la
  cantidad que hay en cada uno. Cuanto mejor sea la fuente (tenga una
  mejor puntuación de la función de evaluación), más probable será que
  las abejas observadoras se dirijan a ella.
\item
  Fase de buscadoras: Las abejas recolectoras que hayan pasado un número
  de iteraciones límite sin encontrar mejores fuentes de alimento en su
  vecindario, darán el vecindario como exhausto y comenzarán a buscar
  nuevas soluciones aleatoriamente, moviéndose por el resto del espacio.
\end{itemize}

La ventaja de este algoritmo respecto a PSO, a pesar de estar basado en
ideas relativamente similares, es que ABC se puede aplicar de forma más
sencilla a problemas con una codificación distinta. Dado que aquí no
tenemos el concepto de velocidad para modificar las soluciones, si no
que exploramos el vecindario de las buenas soluciones por medio de una
búsqueda local, no necesitamos que el espacio de soluciones esa un
subconjunto convexo de \(\mathbb{R}^n\), nos es suficiente con definir
un operador que nos genere soluciones vecinas a una dada.

Una vez hemos resumido el funcionamiento básico de los tres modelos,
vamos a hacer un pequeño estudio comparativo entre las tres técnicas

\hypertarget{impacto-de-las-tres-tuxe9cnicas-en-trabajos-de-investigaciuxf3n}{%
\section{Impacto de las tres técnicas en trabajos de
investigación}\label{impacto-de-las-tres-tuxe9cnicas-en-trabajos-de-investigaciuxf3n}}

En este apartado, vamos a realizar una pequeña comparación en el impacto
que estas tres técnicas han tenido en el ámbito científico, viendo la
cantidad de publicaciones relacionadas con cada temática que se han
publicado.

Para cada uno de los algoritmos, se ha buscado utilizando Scopus el
número de publicaciones que contienen en su título, \emph{abstract} o
palabras clave las palabras \enquote{genetic algorithm},
\enquote{particle swarm optimization} y \enquote{artificial bee colony},
respectivamente. Se han recopilado los datos entre el año 1990 y la
actualidad. El primer artículo sobre algoritmos genéticos registrado en
la plataforma tiene fecha de 1972. Hemos excluido los años anteriores a
1990 debido a que existen pocas publicaciones hasta esa fecha, y
sólamente sobre algoritmos genéticos, por lo que la gráfica presentaba
muy poca información en su parte izquierda.

Mostramos los datos a continuación

\includegraphics{entrega_files/figure-latex/unnamed-chunk-1-1.pdf}

Lo primero que podemos observar es que el número de publicaciones sobre
algoritmos genéticos sobrepasa en una cantidad muy considerable a los
otros dos paradigmas. Esto puede deberse, en primer lugar, a la
generalidad del primer enfoque con respecto a los otros dos. Mientras
que PSO y ABC son algoritmos más o menos concretos, en los que el
funcionamiento del mismo está más o menos bien determinado, y las
modificaciones que pueden hacerse son reducidas, para el algoritmo
genético aparece mucha más variabilidad. Una propuesta de operador de
cruce o una estrategia de selección novedosa pueden ser motivo de
publicación, cosa que no ocurre en los otros dos ámbitos, que están
mucho más definidos. Además, el concepto de algoritmo genético es más
maduro, por lo que ha habido más tiempo para investigar al respecto y
por tanto es más probable que aparezcan nuevas ideas y aplicaciones del
mismo.

En los tres casos podemos observar una tendencia creciente en el número
de publicaciones anuales, lo que sugiere que la investigación en estas
áreas no está terminada. En los algoritmo genéticos puede apreciarse
entre 2010 y 2015 una desaceleración en el número anual de
publicaciones, lo cual podía sugerir que el área de investigación estaba
exhausta, pero en los últimos años ha experimentado un repunte, lo cual
indica que sigue siendo un área interesante. PSO parece estar
experimentando una subida equivalente a la de GA entre el 1995 y el
2005, y ABC, al ser el modelo más nuevo, está comenzando a tomar fuerza.

Hablando del número total de citas, podemos observar la abismal
diferencia entre los tres modelos:

\begin{longtable}[]{@{}rrr@{}}
\caption{Número total de citas}\tabularnewline
\toprule
GA & PSO & ABC\tabularnewline
\midrule
\endfirsthead
\toprule
GA & PSO & ABC\tabularnewline
\midrule
\endhead
189045 & 64845 & 6310\tabularnewline
\bottomrule
\end{longtable}

Lo cual reafirma la información que extrajimos en la gráfica. Es posible
que debido a la diferencia en la antigüedad entre las tres
aproximaciones llegue un momento en el cual haya aproximadamente el
mismo número de publicaciones en las tres áreas, pero parece poco
probable teniendo en cuenta la generalidad de la primera en comparación
con las otras dos.

\end{document}
